\documentclass[letterpaper,landscape,11pt]{article}
\usepackage{../../_latex/usenix2019}
\usepackage{epsfig}
\usepackage{amsmath}
\usepackage{amsthm}
\usepackage{amssymb}
\usepackage{csquotes}
\usepackage{enumitem}
\usepackage{fancyhdr}
\usepackage{hyperref}
\usepackage[lmargin=0.25in,tmargin=0.25in]{geometry}
\usepackage{cleveref}
\usepackage{multicol}
\usepackage{tikz}
\usepackage{pgfplots}
\usepackage{lipsum}

\setlength{\textheight}{8in}
\setlength{\textwidth}{10.5in}

\hypersetup{
	colorlinks=true,
	urlcolor=blue,
}

\newtheorem{theorem}{Theorem}
\newtheorem{lemma}{Lemma}
\crefname{lemma}{Lemma}{Lemmas}

\usetikzlibrary{automata,positioning}

% Some useful macros %
\newcommand{\set}[1]{\left \{ #1 \right \}}                     % Set notation: { ... }
\newcommand{\expect}[1]{\operatorname{E}\left[\,#1\,\right]}    % Expectation: E[ blah ]
\newcommand{\notni}{\not\owns}

\newcommand{\hideFromPandoc}[1]{#1}
\let\Begin\begin
\let\End\end


\begin{document}
%don't want date printed
\date{}

\Begin{multicols}{4}

\section{Prerequisite Definitions}
\emph{Alphabets} $\Sigma$, and $\Gamma$ are finite nonempty sets of symbols.

A \emph{string} is a finite sequence of zero or more symbols from an alphabet.

$\Sigma^\star$ is the set of all strings over alphabet $\Sigma$.

$\epsilon$ is the empty string and cannot be in $\Sigma$.

A \emph{problem} is a mapping from strings to strings.

A \emph{decision problem} is a problem whose output is yes/no (or often accept/reject).

A decision problem be thought of as the set of all strings for which the function outputs ``accept''.

A \emph{language} is a set of strings, so any set $S \subseteq \Sigma^\star$ is a language, even $\emptyset$. Thus, decision problems are equivalent to languages.

\section{Regular Languages}
$L(M)$ is the language accepted by machine $M$.

A deterministic finite automaton is a 5-tuple $M = (Q, \Sigma, \delta, q_0, F)$, where
\begin{itemize}
	\item $Q$ is a finite set of states,
	\item $\Sigma$ is an alphabet,
	\item $\delta : Q \times \Sigma \rightarrow Q$ is a transition function describing its transitions and labels,
	\item $q_0 \in Q$ is the starting state, and
	\item $F \subseteq Q$ is a set of accepting states.
\end{itemize}
If $\delta$ is not fully specified, we assume an implicit transition to an \emph{error state}.

A deterministic finite automaton $M$ accepts input string $w = w_1w_2 \dots w_n$ ($w_i \in \Sigma$) if there exists a sequence of states $r_0, r_1, r_2, \dots, r_n$ ($r_i \in Q$) such that
\begin{itemize}
	\item $r_0 = q_0$,
	\item for all $i \in \{1, \dots, n\}$, $r_i = \delta(r_{i-1}, w_i)$, and
	\item $r_n \in F$.
\end{itemize}
$r_0, r_1, r_2, \dots, r_n$ are the sequence of states visited during the machine's computation.

A non-deterministic finite automaton is a 5-tuple $M = (Q, \Sigma, \delta, q_0, F)$, where
\begin{itemize}
	\item $Q, \Sigma, q_0, F$ are the same as a deterministic finite automaton's, and
	\item $\delta : Q \times (\Sigma \cup \{\epsilon\}) \rightarrow 2^Q$.
\end{itemize}

A non-deterministic finite automaton accepts the string $w = w_1w_2 \dots w_n$ ($w_i \in \Sigma$) if there exist a string $y = y_1y_2 \dots y_m$ ($y_i \in \Sigma \cup \{\epsilon\}$) and a sequence $r = r_0, r_1, \dots, r_n$ ($r_i \in Q$) such that
\begin{itemize}
	\item $w = y_1 \circ y_2 \circ \cdots \circ y_m$ (i.e. $y$ is $w$ with some $\epsilon$ inserted),
	\item $r_0 = q_0$,
	\item for all $i = \{1, \dots, m\}$, $r_i \in \delta(r_{i-1}, q_i)$, and
	\item $r_m \in F$.
\end{itemize}

The \emph{$\epsilon$-closure} for any set $S \subseteq Q$ is denoted $E(S)$, which is the set of all states in $Q$ that can be reachable by following any number of $\epsilon$-transition.

\begin{theorem}
	A non-deterministic finite automaton can be converted to an equivalent deterministic finite automaton.
\end{theorem}

A \emph{regular language} is any language accepted by some finite automaton. The set of all regular languages is called the \emph{class of regular languages}.

\begin{theorem}
	Regular languages are closed under
	\begin{itemize}
		\item Concatenation $L_1 \circ L_2 = \{x \circ y : x \in L_1 \text{ and } y \in L_2\}$. Note: $L_1 \not\subseteq L_1 \circ L_2$.
		\item Union $L_1 \cup L_2 = \{x : x \in L_1 \text{ or } x \in L_2\}$.
		\item Intersection $L_1 \cap L_2 = \{x : x \in L_1 \text{ and } x \in L_2\}$.
		\item Complement $\overline{L} = \Sigma^\star \setminus L = \{x : x \notin L\}$.
		\item Star $L^\star = \{x_1 \circ x_2 \circ \cdots \circ x_k : x_i \in L \text{ and } k \geq 0\}$.
	\end{itemize}
\end{theorem}

$R$ is a regular expression if $R$ is
\begin{itemize}
	\item $a \in \Sigma$,
	\item $\epsilon$,
	\item $\emptyset$,
	\item $R_1 \cup R_2$, or $R_1 | R_2$,
	\item $R_1 \circ R_2$, or $R_1R_2$,
	\item $R_1^\star$,
	\item Shorthand: $\Sigma = (a_1 | a_2 | \dots | a_k)$, $a_i \in \Sigma$,
\end{itemize}
where $R_i$ is a regular expression.

Identities of Regular Languages
\begin{itemize}
	\item $\emptyset \cup R = R \cup \emptyset = R$
	\item $\emptyset \circ R = R \circ \emptyset = \emptyset$
	\item $\epsilon \circ R = R \circ \epsilon = R$
	\item $\epsilon^\star = \epsilon$
	\item $\emptyset^\star = \emptyset$
	\item $\emptyset \cup R \circ R^\star = R \circ R^\star \cup \epsilon = R^\star$
	\item $(a | b)^\star = (a^\star | b^\star)^\star = (a^\star b^\star)^\star = (a^\star | b)^\star = (a | b^\star)^\star = a^\star(ba^\star)^\star = b^\star(ab^\star)^\star$
\end{itemize}

\begin{theorem}
	Languages accepted by DFAs = languages accepted by NFAs = regular languages
\end{theorem}

\begin{theorem}
	If $L$ is a finite language, $L$ is regular.
\end{theorem}

If a computation path of any finite automaton is longer than the number of states it has, there must be a cycle in that computation path.

\begin{lemma}[Pumping Lemma]
	Every regular language satisfies the pumping condition.
\end{lemma}

\emph{Pumping condition}: There exists an integer $p$ such that for every string $w \in L$, with $|w| \geq p$, there exist strings $x, y, z \in \Sigma^\star$ with $w = xyz, y \neq \epsilon, |xy| \leq p$ such that for all $i \geq 0$, $xy^iz \in L$.

\emph{Negation of pumping condition}: For all integers $p$, there exists a string $w \in L$, with $|w| \geq p$, for all $x, y, z \in \Sigma^\star$ with $w = xyz, y \neq \epsilon, |xy| \leq p$, there exists $i \geq 0, i \neq 1$ such that $xy^iz \notin L$.

Limitations of finite automata:
\begin{itemize}
	\item Only read input once, left to right.
	\item Only finite memory.
\end{itemize}

\section{Context-Free Languages}
A pushdown automaton is a 6-tuple $M = (Q, \Sigma, \Gamma, \delta, q_0, F)$, where
\begin{itemize}
	\item $Q$ is a finite set of states,
	\item $\Sigma$ is its input alphabet,
	\item $\Gamma$ is its stack alphabet,
	\item $\delta : Q \times (\Sigma \cup \{\epsilon\}) \times (\Gamma \cup \{\epsilon\}) \rightarrow 2^{Q \times (\Gamma \cup \{\epsilon\})}$ is its transition function,
	\item $q_0 \in  Q$ is its starting state, and
	\item $F \subseteq Q$ is a finite set of accepting states.
\end{itemize}

Labels: $a, b \rightarrow c$: if input symbol is $a$, and top of stack is $b$, pop it and push $c$. In other words, input symbol read, stack symbol popped $\rightarrow$ stack symbol pushed, e.g. $0, \epsilon \rightarrow \$$.

Suppose $u, v, w$ are strings of variables and terminals, and there is a rule $A \rightarrow w$. From the string $uAv$, we can obtain $uwv$. We write $uAv \rightarrow uwv$, and say $uAv$ yields $uwv$.

If $u_1 \rightarrow u_2 \rightarrow \dots \rightarrow u_k$, then $u_1 \rightarrow^\star u_k$, or $u_1$ derives $u_k$. There must be a finite number of arrows between $u_1$ and $u_k$.

Given a grammar $G$, the language derived by the grammar is $L(G) = \{w \in \Sigma^\star : S \rightarrow^\star w \text{ and } S \text{ is the start variable}\}$

\emph{Context-free grammar:} the lhs of rules is a single variable, rhs is any string of variables and terminals. A \emph{context-free language} is one that can be derived from a context-free grammar. An example context-free grammar is $G = (V, \Sigma, R, \langle \texttt{EXPR} \rangle)$, where $V = \{\langle \texttt{EXPR} \rangle, \langle \texttt{TERM} \rangle, \langle \texttt{FACTOR} \rangle\}$, $\Sigma = \{a, +, \times, (, )\}$, and $R = \{\langle \texttt{EXPR} \rangle \rightarrow \langle \texttt{EXPR} \rangle + \langle \texttt{TERM} \rangle | \langle \texttt{TERM} \rangle, \langle \texttt{TERM} \rangle \rightarrow \langle \texttt{TERM} \rangle \times \langle \texttt{FACTOR} \rangle | \langle \texttt{FACTOR} \rangle, \langle \texttt{FACTOR} \rangle \rightarrow (\langle \texttt{EXPR} \rangle)\}$.

A \emph{left-most derivation} is a sequence $S \rightarrow u_1 \rightarrow u_2 \rightarrow \dots \rightarrow u_k \rightarrow w$ where each step applies a rule to the left-most variable. A grammar is \emph{ambiguous} when it has multiple left-most derivations for the same string.

\begin{theorem}
	A language $L$ is recognized by a pushdown automaton iff $L$ is described by a context-free grammar.
\end{theorem}

\begin{theorem}
	Context-free languages are closed under union, concatenation, star.
\end{theorem}

\section{Recognizable Languages}
Differences from previous models
\begin{itemize}
	\item The input is written on tape.
	\item It can write to the tape.
	\item It can move left and right on tape.
	\item It halts immediately when it reaches an accepting or rejecting state. The rejecting state must exist but may not be shown.
\end{itemize}

A deterministic Turing machine is a 7-tuple $M = (Q, \Sigma, \Gamma, \delta, q_0, q_{accept}, q_{reject})$, where
\begin{itemize}
	\item $Q$ is its finite non-empty set of states,
	\item $\Sigma$ is its input alphabet,
	\item $\Gamma$ is its tape alphabet ($\Sigma \subset \Gamma$ and $\text{\textvisiblespace } \in \Gamma \setminus \Sigma$),
	\item $\delta : Q \times \Gamma \rightarrow Q \times \Gamma \times \{L, R\}$ is its transition function,
	\item $q_0 \in Q$ is its starting state,
	\item $q_{accept} \in Q$ is its accepting state, and
	\item $q_{reject} \in Q$ is its rejecting state ($q_{reject} \neq q_{accept}$).
\end{itemize}

Labels: $a \rightarrow b, R$: if tape symbol is $a$, write $b$ and move head right. $a \rightarrow R$: if tape symbol is $a$, move head right. $a, b, c \rightarrow R$: if tape symbol is $a, b$, or $c$, move head right.

On input $x$, a Turing machine can (1) accept, (2) reject, or (3) run in an infinite loop.

The language \emph{recognized} by a Turing machine $M$ is $L(M) = \{x : \text{on input } x, M \text{ halts in } q_{accept}\}$. A language is \emph{recognizable} if there exists a Turing machine which recognizes it.

Regular languages $\subseteq$ context-free languages $\subseteq$ decidable languages $\subseteq$ recognizable languages

A \emph{configuration} is a way to describe the entire state of the Turing machine. It is a string $aqb$ where $a \in \Gamma^\star, q \in Q, b \in \Gamma^\star$, which indicates that $q$ is the current state of the Turing machine, the tape content currently is $ab$ and its head is currently pointing at the first symbol of $b$. Any Turing machine halts if its configuration is of the form $aq_{accept}b$, or $aq_{reject}b$ for any $ab$. Config($i$) uniquely determines Config($i + 1$).

\begin{theorem}
	Every $k$-tape Turing machine has an equivalent single tape Turing machine.
\end{theorem}
If the alphabet of the multitape Turing machine is $\Gamma$, we can make the single tape Turing machine's alphabet $(\Gamma \cup \{\#\}) \times \{\texttt{normal}, \texttt{bold}\}$.

A non-deterministic Turing machine is a 7-tuple $M = (Q, \Sigma, \Gamma, \delta, q_0, q_{accept}, q_{reject})$, where the only difference from a deterministic Turing machine is the transition function $delta : Q \times \Gamma \rightarrow 2^{Q \times \Gamma \times \{L, R\}}$.

A non-deterministic Turing machine accepts its input iff some node in the configuration tree has $q_{accept}$. It does not accept its input iff the configuration tree grows forever (infinite loop) or no node in the tree has $q_{accept}$.

\emph{Acceptance of a non-deterministic Turing machine:} input $w$ is accepted if there exist configurations $c_0, c_1, \dots, c_k$ where
\begin{itemize}
	\item $c_0 = q_{start}w$, and
	\item $c_i \Rightarrow c_{i+1}$ ($c_{i+1}$ is a possible configuration from $c_i$, following the transition function $\delta$).
\end{itemize}

The outcomes could be
\begin{itemize}
	\item $w$ is accepted, i.e. there exists a node in the tree which is an accepting configuration,
	\item $w$ is explicitly rejected, i.e. the tree is finite but no node is an accepting configuration (all leaves are rejecting configurations), or
	\item the non-deterministic Turing machine runs forever on $w$, i.e. the tree is infinite but no node is an accepting configuration (there might be finite branches terminating in a rejecting configuration in the tree).
\end{itemize}

A Turing machien is a \emph{decider} if it halts on all inputs, i.e. it either rejects or accepts all inputs.

\begin{theorem}
	Every non-deterministic Turing machine has an equivalent deterministic Turing machine. If that non-deterministic Turing machine is a decider, there is an equivalent deterministic Turing machine decider.
\end{theorem}

\begin{theorem}
	Recognizable languages are closed under union, intersection, concatenation, star.
\end{theorem}

Implementation level description of a multitape Turing machine for $L = \{x\#x : x \in \{0, 1\}^\star\}$:
\begin{itemize}
	\item Scan the first head to the right until it reads a $\#$. Move right. The second head is still at the start of the second tape.
	\item Repeatedly read symbol from the first tape (reject if the symbol is not $0$ or $1$), write it to the second tape, and move both heads right, until seeing a blank on the first tape.
	\item Move the first head left until a $\#$ is under it. Replace the symbol with a blank (\textvisiblespace).
	\item Move both heads left until they reach the start of their respective tapes (using the $\$$ sign hack to mark the start of the tape).
	\item Repeat until seeing a blank on both tapes.
	\begin{itemize}
		\item If the symbols on the two tapes differ, reject.
		\item Otherwise, move both head right.
	\end{itemize}
\end{itemize}

$\langle O \rangle$ is a string encoding for the object $O$.

\emph{Cardinality of Sets:} two sets $A$ and $B$ have the same \emph{cardinality} if there exists a bijection $f : A \rightarrow B$.

$\mathbb{N} = \{1, 2, 3, \dots\}$ is the set of all natural numbers. A set is \emph{finite} if it has a bijection to \{1..n\} for some natural number $n$. A set is \emph{countably infinite} if it has the same cardinality as $\mathbb{N}$. A set is \emph{countable} or \emph{at most countable} if it is finite or countably infinite.

\begin{lemma}
	Any language $L$ is countable.
\end{lemma}

\begin{lemma}
	The set of all Turing machines is countable.
\end{lemma}

\begin{lemma}
	The set $\mathcal{B}$ of all infinite bit-sequences is not countable.
\end{lemma}

\begin{lemma}
	$2^{\Sigma^\star}$ is uncountable.
\end{lemma}

\section{Reductions}

$A_{TM} = \{\langle M, w \rangle : M \text{ accepts } w\}$ and $HALT_{TM} = \{\langle M, w \rangle : M \text{ halts on input } w\}$ are recognizable but not decidable.

\begin{theorem}
	If $L$ and $\overline{L}$ are recognizable, then $L$ is decidable (and so is $\overline{L}$).
\end{theorem}

\begin{lemma}
	$\overline{A_{TM}}$ is unrecognizable.
\end{lemma}

\emph{Proof template for undecidability via Turing reduction:} Reduce a problem known to be undecidable to that language $L$, usually $A_{TM}$, i.e. $A_{TM} \leq_T L$. Assume a Turing machine decider $R$ for $L$. Construct $S$ that decides $A_{TM}$ using $R$.

\emph{Runtime of a deterministic Turing machine} is a function $f : \mathbb{N} \rightarrow \mathbb{N}$ given by $f(n) = max_{x \in \Sigma^\star, |x| = n} (\text{no. of steps of } M \text{ on input } x)$.

$TIME(t(n)) = \{\text{language } L : \exists \text{deterministic Turing machine that} \\ \text{decides } L \text{ in time } O(t(n))\}$.

$P = \bigcup_{c \geq 0} TIME(n^c)$

$EXP = \bigcup_{k \geq 0} TIME(2^{n^k})$

\begin{theorem}[Time hierarchy theorem]
	If $f : \mathbb{N} \rightarrow \mathbb{N}$ is reasonable and $f = \Omega(n\log n)$ then $TIME(f(n)) \subset TIME(f(n)^2)$.
\end{theorem}

\begin{lemma}
	$P \subset EXP$
\end{lemma}

\emph{Runtime of a non-deterministic Turing machine} is the height of the configuration tree.

$NTIME(t(n)) = \{\text{language } L : \exists \text{ non-deterministic Turing machine that} \\ \text{decides } L \text{ in time } t(n)\}$

$NP = \bigcup_{c > 0} NTIME(n^c)$, i.e. languages for which it is easy to verify membership.

\begin{lemma}
	$P \subseteq NP$
\end{lemma}

\begin{lemma}
	$NP \subseteq EXP$
\end{lemma}

\emph{Verifier-based definition for $L \in NP$:} there exists a deterministic polytime Turing machine $V$ and a constant $c$ such that $L = \{x \in \Sigma^\star : \exists y \in \Sigma^\star, |y| \leq |x|^c, V \text{ accepts } (x, y)\}$.

A function is \emph{polytime computable} if $f : \Sigma^\star \rightarrow \Sigma^\star$ if there exists a Turing machine $M$ that has $x$ as input, runs for time poly($|x|$) and halts with $f(x)$ written on the tape.

$f$ is a \emph{polytime reduction} from language $A$ to language $B$, denoted $A \leq_P B$ if (1) $f(A) \subseteq B$, (2) $f(\overline{A}) \subseteq \overline{B}$, and (3) $f$ is a polytime computable function.

\begin{theorem}
	If $A \leq_P B$ and $B \in P$ then $A \in P$.
\end{theorem}

A language $L$ is \emph{$NP$-hard} if $A \leq_P L$ for all $A \in NP$. A language $L$ is \emph{$NP$-complete} if $L$ is $NP$-hard and $L \in NP$.

\begin{theorem}
	If $P \neq NP$, then there exists language $L$ such that $L \notin NP$-complete, $L \notin P$, and $L \in NP$.
\end{theorem}

\begin{theorem}
	If (1) $B$ is $NP$-complete, (2) $C \in NP$, and (3) $B \leq_P C$, then $C$ is $NP$-complete.
\end{theorem}

\begin{theorem}
	$CLIQUE$ is $NP$-complete
\end{theorem}

\begin{theorem}
	$3SAT \leq_P CLIQUE$
\end{theorem}

\begin{theorem}
	$3SAT \leq_P MAXCLIQUE$
\end{theorem}

Reductions from $3SAT$ often involves \emph{gadgets}:
\begin{itemize}
	\item \emph{Clause gadgets:} for the assignemnt to pick a true literal in each clause (a clique must pick a vertex from each group)
	\item \emph{Variable gadget:} force assignemnt to set each variable either to true or false but not both (a clique cannot pick both $x_i$ and $\overline{x_i}$).
\end{itemize}

\begin{theorem}
	$SAT$ is $NP$-complete via $a$ where
	\begin{gather*}
		C = Q \cup \{\#\} \cup \Gamma \\
		x_{i, j, s} = true \Leftrightarrow cell[i, j] = s \\
		\Phi_{start} = x_{1, 1, \#} \land x_{1, 2, q_{start}} \land x_{1, 3, w_1} \land \dots \\
		\land x_{1, n^k-1, \text{\textvisiblespace}} \land x_{1, n^k, \#} \\
		\Phi_{cell} = \bigwedge\limits_{i, j = 1}^{n^k} \left(\bigvee\limits_{s \in C} x_{i, j, s} \land \right. \\
		\left. \bigwedge\limits_{s, t \in C, 1 \leq i, j \leq n^k} \lnot (x_{i, j, s} \land x_{i, j, t})\right) \\
		\Phi_{moves} = \bigwedge\limits_{i, j \geq n^k} \text{(window}[i, j] \text{ is valid)} \\
		\Phi_{accept} = \bigvee\limits_{1 \leq i, j \leq n^k} x_{i, j, q_{accept}}
	\end{gather*}
\end{theorem}

$coNP = \{\text{language } L : \overline{L} \in NP\}$, i.e. languages for which it is easy to verify non-membership. Machine model for $L \in coNP$ is when $x \in L$, all leaves are accepting configurations; otherwise, when $x \notin L$, there exists one leaf which is a rejecting configuration.

$coNP$-complete $ = \{\text{language } B : B \in coNP, \forall A \in coNP, A \leq_P B\}$.

\begin{theorem}
	$NOSAT$ is $coNP$-complete.
\end{theorem}

\begin{lemma}
	$L \in NP$-complete iff $\overline{L} \in coNP$-complete.
\end{lemma}

\section{Probabilistic Turing Machines}

$RP$, or randomized polynomial time, are the languages $L$ for which there is a \emph{probabilistic} Turing machine that, on input $x$, runs in poly($|x|$) and when $x \in L$, Pr[reaching accept] $\geq \frac{1}{2}$; otherwise, when $x \notin L$, Pr[reaching reject] $ = 1$.

\emph{Second definition for $RP$:} it contains languages $L$ for which there exists a deterministic polytime Turing machine $V$ such that when $x \in L$, for at least half of all $y$ with $|y| \leq \text{poly}(|x|), V$ accepts $(x, y)$; when $x \notin L$, for all $y$ with $|y| \leq \text{poly}(|x|), V$ rejects $(x, y)$.

Contrast with $NP$, where $\forall x \in L$, Pr[reaching accept] $ > 0$, $\forall x \notin L$, Pr[reaching reject] $ = 1$.

\begin{theorem}
	$RP \subseteq NP$
\end{theorem}

$coRP$: $\forall x \in L$, Pr[reaching accept] $ = 1$, $\forall x \notin L$, Pr[reaching reject] $\geq \frac{1}{2}$.

$coNP$: $\forall x \in L$, Pr[reaching accept] $ = 1$, $\forall x \notin L$, Pr[reaching reject] $> 0$.

$BPP$, or bounded error probabilistic polynomial time: $\forall x \in L$, Pr[reaching accept] $\geq \frac{2}{3}$, $\forall x \notin L$, Pr[reaching reject] $\geq \frac{2}{3}$.

\begin{lemma}
	$RP \subseteq BPP$
\end{lemma}

\begin{lemma}
	$coRP \subseteq BPP$
\end{lemma}

\begin{lemma}
	$RP(\frac{1}{2}) = RP(\frac{3}{4})$ (proof via amplification)
\end{lemma}

\begin{lemma}
	$RP$ is closed under composition.
\end{lemma}

\section{Communication Complexity}
Model:
\begin{itemize}
	\item Finite sets $X, Y, Z$
	\item Function $f : X \times Y \rightarrow Z$
	\item Two player, Alice and Bob
	\item Decide on a communication protocol beforehand
	\item Alice has $x \in X$, Bob has $y \in Y$
	\item Goal: collaboratively compute $f(x, y)$ by sending bits back and forth (must end with both side knowing $f(x, y)$)
\end{itemize}

\noindent The \emph{trivial prototol:}
\begin{itemize}
	\item Alice sends $x$ to Bob ($\log |X|$)
	\item Bob computes and sends $z = f(x, y)$ to Alice ($\log |Z|$)
\end{itemize}
Total: $\log |X| + \log |Z|$ or $\log |Y| + \log |Z|$

A \emph{communication protocol} is a binary tree where each node is labelled by either $a_v : X \rightarrow \{L, R\}$ or $b_v : Y \rightarrow \{L, R\}$ and each leaf is labelled by an element of $Z$. The depth of the protocol tree is the maximum number of bits sent by the protocol.

The deterministic communication complexity of a function $f$ is
\begin{align*}
	D(f) & = \min\limits_{\text{tree for } f} \left(\max\limits_{(x, y)}(\text{number of bits}) \right) \\
	& = \min\limits_{\text{tree for } f} (\text{depth of tree})
\end{align*}

\begin{lemma}
	$D(EQ_n) \leq n + 1$
\end{lemma}

A \emph{rectangle} in $X \times Y$ is a set of the form $R = A \times B$ where $A \subseteq X$ and $B \subseteq Y$. $R$ is a rectangle iff $(x, y) \in R \land (x', y') \in R \Leftrightarrow (x, y') \in R \land (x', y) \in R$

\begin{lemma}
	Let $T$ be a protocol tree, $R_v$ be the set of inputs that causes the protocol to arrive at node $v$. Then $R_v$ is a rectangle.
\end{lemma}

A rectangle is called \emph{$f$-monochromatic} if $f(x, y)$ is the same for all $(x, y) \in R$.

Let $R_i \subset X \times Y$ be a rectangle for $i = 1, \dots, k$. The set $\mathcal{R} = \{R_1, \dots, R_k\}$ is called an \emph{$f$-monochromatic partition (into rectangles)} if each $R_i$ is $f$-monochromatic, and each $(x, y) \in X \times Y$ is contained in exactly one $R_i$.

$C^{\text{partition}}(f) = \min \{|\mathcal{R}| : \mathcal{R} \text{ is an }f\text{-monochromatic partition}\}$

\begin{lemma}
	For any protocol tree $T$, the rectangles $\{R : v \text{ is a leaf in } T\}$ are an $f$-monochromatic partition.
\end{lemma}

\begin{lemma}
	$C^{\text{partition}}(f) \leq \min\limits_{\text{protocol tree } T} |\text{number of leaves in } T|$
\end{lemma}

\begin{lemma}
	$D(f) \geq \left \lceil \log_2 C^{\text{partition}}(f) \right \rceil$
\end{lemma}

A \emph{fooling set} $S \subseteq X \times Y$ is a set where all points $(x, y) \in S$ have the same value $f(x, y) = z$, and for any distinct points $(x, y)$ and $(x', y')$ in $S$, either $f(x, y') \neq z$ or $f(x', y) \neq z$.

\begin{lemma}
	$C^{\text{partition}}(f) \geq |S| + 1$, where $S$ is a fooling set for $f$
\end{lemma}

\begin{lemma}
	$D(f) \geq \left \lceil \log_2(|S| + 1) \right \rceil$, where $S$ is a fooling set for $f$
\end{lemma}

\begin{lemma}
	$D(EQ_n) = D(GTE_n) = D(DISJ_n) = n + 1$
\end{lemma}

Model for \emph{non-deterministic communication complexity}:
\begin{itemize}
	\item Function $f : X \times Y \rightarrow Z$ is known to all
	\item Bob does not know $x$, Alice does not know $y$
	\item Alice and Bob do not communicate
	\item Piere tries to force Alice and Bob to accept by sending certificate $z$. How short can $z$ be?
\end{itemize}

$N(f) = \min\limits_{nondet protocol}(\text{length of cert})$. Or, $N(f) = \min\{k\}$ such that there exist $A$ and $B$, for all $x \in X$, $y \in Y$, $f(x, y) = 1 \Rightarrow \exists z \in \{0, 1\}^k, A(x, z) = 1 \land B(y, z) = 1$, $f(x, y) = 0 \Rightarrow \forall z \in \{0, 1\}^k, A(x, z) = 0 \lor B(y, z) = 0$.

\begin{lemma}
	$N(\lnot DISJ_n) \leq \log n$
\end{lemma}

\begin{lemma}
	For all $f$, $D(f) = D(\lnot f)$.
\end{lemma}

\begin{lemma}
	$N(\lnot EQ_n) \leq \log(n) + 1$
\end{lemma}

\begin{lemma}
	Let $S$ be a fooling set where $f(x, y) = 1$ for all $(x, y) \in S$. Then $N(f) \geq \left \lceil \log_2(|S|) \right \rceil$.
\end{lemma}

\begin{lemma}
	$N(EQ_n) \geq n$
\end{lemma}

The set $\mathcal{R} = \{R_1, \dots, R_k\}$ is a cover of the $1$-entries (by rectangles) if (1) each $R_i$ is a rectangle containing only $1$s, and (2) every $(x, y) \in X \times Y$ with $f(x, y) = 1$ is contained in at least one $R_i$.

$C^{1\text{-cover}}(f) = \min \{|\mathcal{R}| : \mathcal{R} \text{ is a cover of the } 1\text{-entries}\}$

$C^{0\text{-cover}}(f) = C^{1\text{-cover}}(\lnot f)$. 

\begin{lemma}
	$C^{\text{partition}}(f) = C^{1\text{-cover}}(f) + C^{0\text{-cover}}(f)$. 
\end{lemma}

\begin{lemma}
	$N(f) = \left \lceil \log_2(C^{1\text{-cover}}(f) ) \right \rceil$
\end{lemma}

\begin{lemma}
	$D(f) \geq N(f)$
\end{lemma}

\begin{lemma}
	$D(\lnot f) = N(f)$
\end{lemma}

\begin{theorem}
	Let $f : X \times Y \rightarrow \{0, 1\}$ be arbitrary, $C_0$ be a cover of the $0$-entries, and $C_1$ be a cover of the $1$-entries. Then $D(f) = O(\log C_0 * \log C_1)$.
\end{theorem}

\begin{lemma}
	$D(f) = O(N(f) * N(\lnot f))$
\end{lemma}

\End{multicols}

\end{document}
