\documentclass[letterpaper,landscape,11pt]{article}
\usepackage{../../_latex/usenix2019}
\usepackage{epsfig}
\usepackage{amsmath}
\usepackage{amsthm}
\usepackage{amssymb}
\usepackage{csquotes}
\usepackage{enumitem}
\usepackage{fancyhdr}
\usepackage{hyperref}
\usepackage[lmargin=0.25in,tmargin=0.25in]{geometry}
\usepackage{cleveref}
\usepackage{multicol}
\usepackage{tikz}
\usepackage{pgfplots}
\usepackage{lipsum}

\setlength{\textheight}{8in}
\setlength{\textwidth}{10.5in}

\hypersetup{
	colorlinks=true,
	urlcolor=blue,
}

\newtheorem{theorem}{Theorem}
\newtheorem{lemma}{Lemma}
\crefname{lemma}{Lemma}{Lemmas}

\usetikzlibrary{automata,positioning}

% Some useful macros %
\newcommand{\set}[1]{\left \{ #1 \right \}}                     % Set notation: { ... }
\newcommand{\expect}[1]{\operatorname{E}\left[\,#1\,\right]}    % Expectation: E[ blah ]
\newcommand{\notni}{\not\owns}

\newcommand{\hideFromPandoc}[1]{#1}
\let\Begin\begin
\let\End\end


\begin{document}
%don't want date printed
\date{}

\Begin{multicols}{4}

\section{Prerequisite Definitions}
\emph{Alphabets} $\Sigma$, and $\Gamma$ are finite nonempty sets of symbols.

A \emph{string} is a finite sequence of zero or more symbols from an alphabet.

$\Sigma^\star$ is the set of all strings over alphabet $\Sigma$.

$\epsilon$ is the empty string and cannot be in $\Sigma$.

A \emph{problem} is a mapping from strings to strings.

A \emph{decision problem} is a problem whose output is yes/no (or often accept/reject).

A decision problem be thought of as the set of all strings for which the function outputs ``accept''.

A \emph{language} is a set of strings, so any set $S \subseteq \Sigma^\star$ is a language, even $\emptyset$. Thus, decision problems are equivalent to languages.

\section{Regular Languages}
$L(M)$ is the language accepted by machine $M$.

A deterministic finite automaton is a 5-tuple $M = (Q, \Sigma, \delta, q_0, F)$, where
\begin{itemize}
	\item $Q$ is a finite set of states,
	\item $\Sigma$ is an alphabet,
	\item $\delta : Q \times \Sigma \rightarrow Q$ is a transition function describing its transitions and labels,
	\item $q_0 \in Q$ is the starting state, and
	\item $F \subseteq Q$ is a set of accepting states.
\end{itemize}
If $\delta$ is not fully specified, we assume an implicit transition to an \emph{error state}.

A deterministic finite automaton $M$ accepts input string $w = w_1w_2 \dots w_n$ ($w_i \in \Sigma^\star$) if there exists a sequence of states $r_0, r_1, r_2, \dots, r_n$ ($r_i \in Q$) such that
\begin{itemize}
	\item $r_0 = q_0$,
	\item for all $i \in \{1, \dots, n\}$, $r_i = \delta(r_{i-1}, w_i)$, and
	\item $r_n \in F$.
\end{itemize}
$r_0, r_1, r_2, \dots, r_n$ are the sequence of states visited during the machine's computation.

A non-deterministic finite automaton is a 5-tuple $M = (Q, \Sigma, \delta, q_0, F)$, where
\begin{itemize}
	\item $Q, \Sigma, q_0, F$ are the same as a deterministic finite automaton's, and
	\item $\delta : Q \times (\Sigma \cup \{\epsilon\}) \rightarrow 2^Q$.
\end{itemize}

A non-deterministic finite automaton accepts the string $w = w_1w_2 \dots w_n$ ($w_i \in \Sigma^\star$) if there exist a string $y = y_1y_2 \dots y_m$ ($y_i \in (\Sigma \cup \{\epsilon\})^\star$) and a sequence $r = r_0, r_1, \dots, r_n$ ($r_i \in Q$) such that
\begin{itemize}
	\item $w = y_1 \circ y_2 \circ \cdots \circ y_m$ (i.e. $y$ is $w$ with some $\epsilon$ inserted),
	\item $r_0 = q_0$,
	\item for all $i = \{1, \dots, m\}$, $r_i \in \delta(r_{i-1}, q_i)$, and
	\item $r_m \in F$.
\end{itemize}

The \emph{$\epsilon$-closure} for any set $S \subseteq Q$ is denoted $E(S)$, which is the set of all states in $Q$ that can be reachable by following any number of $\epsilon$-transition.

\begin{theorem}
	A non-deterministic finite automaton can be converted to an equivalent deterministic finite automaton.
\end{theorem}

A \emph{regular language} is any language accepted by some finite automaton. The set of all regular languages is called the \emph{class of regular languages}.

\begin{theorem}
	Regular languages are closed under
	\begin{itemize}
		\item Concatenation $L_1 \circ L_2 = \{x \circ y : x \in L_1 \text{ and } y \in L_2\}$. Note: $L_1 \not\subseteq L_1 \circ L_2$.
		\item Union $L_1 \cup L_2 = \{x : x \in L_1 \text{ or } x \in L_2\}$.
		\item Intersection $L_1 \cap L_2 = \{x : x \in L_1 \text{ and } x \in L_2\}$.
		\item Complement $\overline{L} = \Sigma^\star \setminus L = \{x : x \notin L\}$.
		\item Star $L^\star = \{x_1 \circ x_2 \circ \cdots \circ x_k : x_i \in L \text{ and } k \geq 0\}$.
	\end{itemize}
\end{theorem}

$R$ is a regular expression if $R$ is
\begin{itemize}
	\item $a \in \Sigma$,
	\item $\epsilon$,
	\item $\emptyset$,
	\item $R_1 \cup R_2$, or $R_1 | R_2$,
	\item $R_1 \circ R_2$, or $R_1R_2$,
	\item $R_1^\star$,
	\item Shorthand: $\Sigma = (a_1 | a_2 | \dots | a_k)$, $a_i \in \Sigma$,
\end{itemize}
where $R_i$ is a regular expression.

\begin{theorem}
	Languages accepted by DFAs = languages accepted by NFAs = regular languages
\end{theorem}

\begin{theorem}
	If $L$ is a finite language, $L$ is regular.
\end{theorem}

If a computation path of any finite automaton is longer than the number of states it has, there must be a cycle in that computation path.

\begin{lemma}[Pumping Lemma]
	Every regular language satisfies the pumping condition.
\end{lemma}

\emph{Pumping condition}: There exists an integer $p$ such that for every string $w \in L$, with $|w| \geq p$, there exist strings $x, y, z \in \Sigma^\star$ with $w = xyz, y \neq \epsilon, |xy| \leq p$ such that for all $i \geq 0$, $xy^iz \in L$.

\emph{Negation of pumping condition}: For all integers $p$, there exists a string $w \in L$, with $|w| \geq p$, for all $x, y, z \in \Sigma^\star$ with $w = xyz, y \neq \epsilon, |xy| \leq p$, there exists $i \geq 0, i \neq 1$ such that $xy^iz \notin L$.

Limitations of finite automata:
\begin{itemize}
	\item Only read input once, left to right.
	\item Only finite memory.
\end{itemize}

\End{multicols}

\end{document}
